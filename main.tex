 
%----------------------------------------------------------------------------------------
%	PACKAGES AND OTHER DOCUMENT CONFIGURATIONS
%----------------------------------------------------------------------------------------

\documentclass[fleqn,10pt]{SelfArx} 
\usepackage{caption}
\DeclareCaptionType{mycapequ}[][List of equations]

\captionsetup[mycapequ]{labelformat=empty}
 \usepackage{ragged2e}
 \usepackage{appendix}
\usepackage{float}
\usepackage{hyperref} 

%----------------------------------------------------------------------------------------
%	COLUMNS
%----------------------------------------------------------------------------------------

\setlength{\columnsep}{0.55cm} % Distance between the two columns of text
\setlength{\fboxrule}{0.75pt} % Width of the border around the abstract

%----------------------------------------------------------------------------------------
%	COLORS
%----------------------------------------------------------------------------------------

\definecolor{color1}{RGB}{0,0,90} % Color of the article title and sections
\definecolor{color2}{RGB}{0,20,20} % Color of the boxes behind the abstract and headings

%----------------------------------------------------------------------------------------
%	HYPERLINKS
%----------------------------------------------------------------------------------------



\hypersetup{hidelinks,colorlinks,breaklinks=true,urlcolor=color2,citecolor=color1,linkcolor=color1,bookmarksopen=false,pdftitle={Title},pdfauthor={Author}}


%----------------------------------------------------------------------------------------
%	ARTICLE INFORMATION
%----------------------------------------------------------------------------------------

\JournalInfo{EDP et cycle cellulaire, 2018} 
\Archive{} 

\PaperTitle{Uniformisation des hypothèses sur l'origine de l'anémie aplasique et sur hématopoïèse cyclique
} 
\Authors{Noémie Blahic, Emilie Mathian } % Author

\Keywords{anémie aplasique - hématopoïèse périodique} % à compléter
\newcommand{\keywordname}{Keywords} 
%----------------------------------------------------------------------------------------
%	ABSTRACT
%----------------------------------------------------------------------------------------

\Abstract{\textbf{\cite{mackey1978unified}}} % Si tu veux on l'écrira à la fin l'intro c'est le )plus dur il faudrait mieux avoir une vision globale avant non?
% 
%----------------------------------------------------------------------------------------

\begin{document}

\flushbottom % Makes all text pages the same height

\maketitle % Print the title and abstract box

\thispagestyle{empty} % Removes page numbering from the first page

%----------------------------------------------------------------------------------------
%	ARTICLE CONTENTS
%----------------------------------------------------------------------------------------

\section*{Introduction} 

\subsection*{Auteurs}
\subsection*{Journal}  
Le journal \textit{Blood} est publié depuis 1946 par la société Américaine d'hématologie. Le fondateur du journal et de cette société est le docteur William Dameshek, connu comme étant l'un des pionnier de la chimiothérapie. Le journal rassemble près de 20000 publications sur les cellules sanguines, la génomique, le système immunitaire ou encore certaines techniques médicales comme la thérapie génique ou la transplantation. On peut recenser également 231 consacrés au cycle cellulaire, dont certains sont dédiés plus précisément à la modélisation mathématique de cette dynamique. Michael C Makey a ainsi publié 5 articles dans ce journal, certainement en raison de l'influence de ce dernier (qui est l'un des plus cités dans le domaine) et de son approche pluri-disciplinaires, bien que foicalisé sur une thématique commune l'hématologie.
\section*{Modèle biologiquet} 



\subsection*{Contexte biologique}
\section*{Modélisation}
\subsection*{Contexte mathématique}
\subsection*{Description du modèle} % + schéma
\subsection*{Hypothèses} % Bio et math

\section*{Analyse}
\subsection*{Analyse théorique}
\subsection*{Analyse numérique}

\section*{Résultats et interprétations }

\section*{Discussion}
\subsection*{Impact}
\subsection*{Critiques}
\subsection*{Suggestion}
%----------------------------------------------------------------------------------------
%	REFERENCE LISt
%----------------------------------------------------------------------------------------
\phantomsection
\bibliographystyle{unsrt}
\bibliography{sample}




-




\end{document}







